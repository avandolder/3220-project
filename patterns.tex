\documentclass{article}
\title{COMP-3220 Project}
\author{Adam Vandolder}
\begin{document}
    \maketitle
    \linespread{1.6}
    \selectfont
    \section{Design Patterns}
    \subsection{Polymorphism Pattern}
    The Dots and Boxes game, as given in the project description, has a fairly
    straightforward use-case of the Polymorphism pattern when it comes to having
    both human and computer players. Essentially, you need to create an abstract
    (a.k.a polymorphic) interface that represents a generic player. You then
    have separate classes for the human and computer players, both of which
    implement that interface. The game can then holds two references to the
    player interface, so from its perspective all players, regardless of whether
    they are a human or a computer, are the same. This would allow for much
    simpler and straightforward game functions, because you would otherwise need
    to keep track of which player was a human and which was a computer and be
    constantly branching based on their type in order to call the correct player
    functions.
    \newpage
    \subsection{Creator Pattern}
    Creator is a GRASP pattern designed to solve the problem of deciding who is
    responsible for creating new instances of a class. Creator is a fundamental
    concern of object-oriented programming, and as such should be considered
    whenever you need to instantiate new objects. When designing our class
    diagram, Creator should be on your mind when creating aggregations and
    compositions. When following the Creator pattern, class A should be the
    creator of class B if it meets any of the following criteria:
    \begin{itemize}
        \item A compositely aggregates B
        \item A records B
        \item A closely uses B
        \item or if A is an Expert with respect to creating B
    \end{itemize}
    It's useful because, if done well, it can help with maintaining low coupling
    and high cohesion between your classes, as well as improving encapsulation
    and reusability.\\
    One example of how the Creator pattern can be used is when considering a
    class structure for a board game. Lets say you have classes representing the
    overall game, the board, and individual spaces that make up the board. Since
    the spaces make up the board (i.e. the board aggregates the spaces),
    following the Creator principle would lead you to make the board create the
    spaces, not the game class. This would, ultimately, allow you to change how
    you represent the board spaces without having to alter anything outside of
    the board class, decreasing coupling and improving encapsulation.
    \newpage
    \subsection{Observer Pattern}
    The Observer pattern is one of the Gang of Four design patterns, that was
    created in order to solve the problem of wanting multiple disparate objects
    to be able to respond to events from a single central publisher object in a
    lowly coupled way. It accomplishes this by having a `listener' interface
    that all objects which wish to listen to the publisher must implement. You
    then add a method on the publisher that allows it to dynamically register
    various listeners, which will then be notified when a corresponding event
    occurs.  This is very useful, because it allows your publisher and listener
    classes to be almost entirely decoupled from each other, even though they
    are directly communicating.\\
    A basic example of the Observer pattern is an event-based windowing manager,
    wherein the manager is the publisher and the individual windows are all
    listeners that will be notified (and often have their thread awoken) when an
    event (i.e. window resizing or redrawing) is fired.\\
    The Observer pattern is often featured as a key part of
    Model-View-Controller architectures.
    \newpage
    \subsection{Decorator Pattern}
    The Decorator pattern is one of the Gang of Four design patterns that allows
    additional behaviour to be added onto a single object of a class
    dynamically, without needing to modify the class as a whole. This is
    accomplished by creating a new decorator class that wraps around the
    original class, usually by creating a new abstract subclass of the original
    class which takes in a concrete object of the original class, and then
    creating subclass of the decorator that add the new behaviour on top of the
    concrete class it is wrapping. In this way, you can add additional features
    to any object without having to create more complex, interweaving
    inheritance trees. Since the individual behaviours can be split up into
    their own classes, the Decorator pattern increases cohesion and lowers
    coupling.\\
    An example of the Decorator pattern can be seen when designing an
    object-oriented windowing system. If you wish to add additional features on
    to a basic window class, for example scroll bars or a menu bar, then
    following a naive OOP approach you would likely make each of these features
    available in their own subclass. However, if you wanted to create a window
    with all of these features, the number of subclasses you would have to
    create rapidly multiplies, resulting in unwieldy inheritance trees and
    producing class names such as
    VerticalAndHorizontalScrollBarAndMenuBarWindow. If you were to split these
    functionalities into separate decorators, though, you could then keep your
    simple and straightforward class diagram and create new windows by creating
    a simple window and then adding on the decorators you desired.
    \newpage
    \subsection{Singleton Pattern}
    The Singleton pattern is one of the Gang of Four creational design patterns.
    It is used in order to restrict the creation of a given class to only one
    object (called the ``singleton''). This is useful when trying to coordinate
    state or actions across the entirety of a system. It is often created by
    making the constructor of the singletons' class private or otherwise
    inaccessible, and forcing other classes to call a static member function in
    order to receive the singleton instance.\\
    Singletons are often used in order to facilitate the use of another design
    pattern; for instance, the Factory and Builder patterns can use Singletons
    in order to limit the amount of memory and resources being used in object
    creation.\\
    Using the Singleton pattern is often viewed as a preferable alternative to
    creating global variables, as Singletons allow the same instance to be
    available across the entire system, like global variables, while not
    polluting the global or containing namespace with additional variables.
    Singletons also allow for lazy allocation and initialization, unlike global
    variables.
    \newpage
    \section{Dots and Boxes}
\end{document}
