\documentclass{article}
\title{Dots and Boxes Design Patterns}
\author{Adam Vandolder}
\begin{document}
    \maketitle
    \section{Polymorphism}
    The Dots and Boxes game, as given in the project description, has a fairly
    straightforward use-case of the Polymorphism pattern when it comes to having
    both human and computer players. Essentially, you need to create an abstract
    (a.k.a polymorphic) interface that represents a generic player. You then
    have separate classes for the human and computer players, both of which
    implement that interface. The game can then holds two references to the
    player interface, so from its perspective all players, regardless of whether
    they are a human or a computer, are the same. This would allow for much 
    simpler and straightforward game functions, because you would otherwise need
    to keep track of which player was a human and which was a computer and be
    constantly branching based on their type in order to call the correct player
    functions.
    \newpage
    \section{Creator}
    Creator is a GRASP pattern designed to solve the problem of deciding who is 
    responsible for creating new instances of a class. Creator is a fundamental
    concern of object-oriented programming, and as such should be considered
    whenever you need to instantiate new objects. When designing our class
    diagram, Creator should be on your mind when creating aggregations and
    compositions. When following the Creator pattern, class A should be the
    creator of class B if it meets any of the following criteria:
    \begin{itemize}
        \item A compositely aggregates B
        \item A records B
        \item A closely uses B
        \item or if A is an Expert with respect to creating B
    \end{itemize}
    It's useful because, if done well, it can help with maintaining low coupling
    and high cohesion between your classes, as well as improving encapsulation
    and reusability. One example of how the Creator pattern

\end{document}
